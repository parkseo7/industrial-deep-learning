\documentclass[12pt]{article}
\usepackage[utf8]{inputenc}

% PACKAGES
\usepackage[margin=1.0in]{geometry}
\usepackage{fancyhdr, etoolbox}
\usepackage{amsmath} 
\usepackage{titlesec}
\usepackage[colorlinks=true, linkcolor=blue]{hyperref}
\usepackage{parskip}
\usepackage{setspace}
\usepackage{listings} % Allows code lines (verbatim, lstlisting)
\usepackage{xcolor} % Colour
\usepackage{enumerate}
\usepackage{multirow} % Enables multicolumn
\usepackage{longtable}

\usepackage{subfiles} % Allows subfile docs
\usepackage{blindtext}

% Fancy header formatting
\pagestyle{fancy}
\setlength{\headheight}{16pt}

\fancyhf{}
\renewcommand{\sectionmark}[1]{ \markright{#1}{} }
\fancyhead[LE,RO]{\thepage}
\fancyhead[RE]{\textit{\nouppercase{\rightmark}}}
\fancyhead[LO]{\textit{\nouppercase{\rightmark}}}

% Section font sizes
\titleformat*{\section}{\Large\bfseries}
\titleformat*{\subsection}{\large\bfseries}
\titleformat*{\subsubsection}{\large}

% Invisible section
\newcommand\invisiblesection[1]{%
  \refstepcounter{section}%
  \addcontentsline{toc}{section}{\protect\numberline{\thesection}#1}%
  \sectionmark{#1}}
  
% New commands:
\newcommand{\code}[1]{\texttt{#1}}
\newcommand{\filestr}[1]{$\langle \text{#1} \rangle$}


\title{Git Cheat Sheet}
\author{Daniel Park}

\begin{document}

\maketitle
\tableofcontents

\pagebreak

\section{Preliminaries}

\begin{center}
    \begin{tabular}{|p{0.25\textwidth}|p{0.7\textwidth}|}
    \hline
        \multicolumn{1}{|c|}{Step} & \multicolumn{1}{c|}{Description} \\
    \hline
        Installing VS Code & Open up VSCode. Go to `Shell Command: Install `code command ...' to set up the environment variable. This enables the command \code{code .} \\
    \hline
        \code{command -flag \filestr{dir}} & Every terminal and git command follows the given format. All file and folder names \filestr{dir} must be in quotations. \\
    \hline
        Configure VS Code & Run \code{git config --global core.editor "code --wait} in git bash. This will configure VS Code as the default git editor over VIM. \\
    \hline
    
    \end{tabular}
    
\end{center}
\section{Terminal commands}

\begin{center}
    \begin{tabular}{|l|l|}
    \hline
        \multicolumn{1}{|c|}{Command} & \multicolumn{1}{c|}{Description} \\
    \hline
         \code{ls} & Lists contents of current directory \\
         \code{ls \filestr{folder}} & Lists contents of \filestr{folder} \\
         \code{ls -a} & Lists all directories and files, including hidden ones \\
    \hline
        \code{start .} & Opens up the file explorer in the current directory \\
        \code{open .} & The \code{start} command on Mac \\
    \hline
        \code{pwd} & Shows the current directory location \\
    \hline
        \code{cd \filestr{folder}} & Changes your current directory to \filestr{folder} \\
        \code{cd ~} & Go back to home directory. Note that ~ denotes home directory \\
        \code{cd ..} & Move backwards to the parent folder of current directory \\
    \hline
        \code{clear} & Clears the terminal \\
    \hline
        \code{code .} & Opens the current directory in VSCode \\
    \hline
        \code{touch \filestr{file}} & Creates a file called \filestr{file} in the current directory \\
        \code{touch \filestr{file1} \filestr{file2}} & Creats files called \filestr{file1} and \filestr{file2} in the current directory \\
        \code{mkdir \filestr{folder}} & Creates a folder called \filestr{folder} \\
    \hline
        \code{rm \filestr{file}} & Deletes the file called \filestr{file} \\
        \code{rm -rf \filestr{folder}} & Deletes the folder \filestr{folder} \\
    \hline
    \end{tabular}
\end{center}

\pagebreak

\section{Git commands}

Git structure: 
\begin{center}
    Working directory $\overset{\code{git add}}{\longrightarrow}$ Staging area $\overset{\code{git commit}}{\longrightarrow}$ Repository
\end{center}

Git GUI: \href{https://www.gitkraken.com}{Git Kraken}. Use the GUI to copy commit hashes.

Resources:
\begin{itemize}
    \item 
    Full documentation: \href{https://git-scm.com/docs/}{git-scm.com/docs}
    \item
    Book on git: \href{https://git-scm.com/book/en/v2}{git-scm.com/book}
\end{itemize}

Git ignore template: \href{https://www.toptal.com/developers/gitignore}{git ignore io}
\begin{itemize}
    \item 
    Use \code{touch `.gitignore'} to create the file.
    \item
    Set up the .gitignore file first before committing files. Otherwise you must remove the files to be ignored from the cache.
\end{itemize}

\begin{center}
    \begin{longtable}{|p{0.25\textwidth}|p{0.7\textwidth}|}
    \hline
        \multicolumn{1}{|c|}{Command} & \multicolumn{1}{c|}{Description} \\
    \hline
        \code{git init} & Initializes the repository in the current directory. Creates a hidden folder called `.git'. \\
    \hline
        \code{git status} & Shows the status of the current git repository. \\
    \hline 
        \code{git add \filestr{file}} & Adds \filestr{file} to the staging area. \\
        \code{git add .} & Adds all changed files to the staging area. \\
    \hline
        \code{git commit} & Uploads all staged files to the git repository. Creates a commit. \\
        \code{git commit -m \filestr{msg}} & Creates a commit with the message \filestr{msg}. \\
        \code{git commit -{}-amend} & Updates the previous commit with all currently staged changes. \\
    \hline
        \code{git log} & Shows the history of all commits made in the repo. \\
        \code{git log --oneline} & Condenses the log into abbreviated hashes and the header line of each commit message. \\
    \hline
        \code{git branch} & Lists all branches. Active branch has an asterisk (*). \\
        \code{git branch \filestr{branch}} & Creates a new branch \filestr{branch} with the parent commit being where HEAD is on. Multiple branches share the same parent commit. \\
    \hline
        \code{git switch \filestr{branch}} & Switches the active branch to \filestr{branch}. All unstaged changes are lost. \\
        \code{git checkout \filestr{branch}} & A generalized command of \code{git switch} and \code{git restore}. \\
        \code{git switch -c \filestr{branch}} & Creates a new branch and switches you over to it. \\
        \code{git checkout \filestr{branch}} & Does the same thing as \code{git switch -c \filestr{branch}}. \\
    \hline
        \code{git branch -d \filestr{branch}} & Deletes the branch \filestr{branch}. Must be full merged before doing so. \\
        \code{git branch -D \filestr{branch}} & Deletes the branch \filestr{branch} unconditionally. \\
    \hline
        \code{git branch -m \filestr{branch}} & Renames the current branch name to \filestr{branch}. \\
    \hline
        \code{git merge \filestr{branch}} & Merges \filestr{branch} to the active branch. If there is a merge conflict, it must be resolved manually. After, you must stage and commit these resolved changes. \\
    \hline
        \code{git diff} & Shows all unstaged changes since the last commit. \\
        \code{git diff HEAD} & Lists all changes in the working tree since your last commit. Includes both staged and unstaged changes. \\
        \code{git diff \filestr{branch1} \filestr{branch2}} & Lists all changes between the tips of \filestr{branch1} and \filestr{branch2}. \\
        \code{git diff \filestr{commit1} \filestr{commit2}} & Lists all changes between the commit hashes \filestr{commit1} and \filestr{commit2}. \\
    \hline
        \code{git stash (save)} & Saves all uncommitted changes to the stash. \\
        \code{git stash pop} & Removes the most recently stashed items and reapplies them to your working directory. \\
        \code{git stash clear} & Clears all stashes. \\
    \hline
        \code{git checkout HEAD \filestr{file}} & Discard any changes made in that file, reverting it back to HEAD. \\
        \code{git checkout HEAD~\filestr{k}} & Detach HEAD $k$ commits back. To re-attach HEAD, use the \code{git switch} command. \\
        \code{git restore \filestr{file}} & Alternative command to \code{git checkout}. \\
        \code{git restore -{}-source \filestr{commit} \filestr{file}} & Reverts \filestr{file} to its previous state at the commit \filestr{commit}. \\
        \code{git restore -{}-staged \filestr{file}} & Unstages the file \filestr{file}. \\
    \hline
        \code{git reset \filestr{commit}} & Resets the repo back to a previous commit. All changes persist as unstaged changes. \\
        \code{git reset -{}-hard \filestr{commit}} & Implements git reset, and all changes are removed from your working directory. \\
        \code{git revert} & Undo the changes as a new commit. Does not delete prior commits made. \\
    \hline
    \end{longtable}
\end{center}


\end{document}
